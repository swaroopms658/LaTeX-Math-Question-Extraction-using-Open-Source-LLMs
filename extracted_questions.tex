Extract only the math questions from the following text.
Output the questions strictly in LaTeX math notation format.
Do NOT add any explanations or other text.

Text:
In section 30.3, we have introduced the concept of conditional probability which has been used
in multiplication theorem of probability. In this section, we will obtain a formula for finding the
conditional probability.
If A and B are two events associated with a random experiment, then
P(AnB)=P(A)P(B/A),ifP(A)7^0
P(AnB) = P(B)P(A/6), iiP{B)^0
P (A n B)
[Multiplication theorem]
or.
P (A n B)
P{A/B) =
and P(B/A) =
P(A)
30.5.1
PROPERTIES OF CONDITIONAL PROBABILITY
Following are some properties of conditional probability which are stated and proved as
theorems.
P(B)
THEOREM 1
Let A and B be two events associated luith sample space S, then 0 < P (A/B) < 1.
PROOF
We know that
A n B c B => P (A n B) < P (B) => P (A n B)
[V P(B)>0]
<1, ifP(B)^0
P(B)
P(AnB)
Also,
P (A n B) > 0 and P (B) > 0. Therefore,
>0
P(B)
P (A n B)
Thus, we obtain:
0
<
<1
or, 0 <P (A/B) <l.Hence,0 <P (A/B) <1
P(B)
Q.E.D.
THEOREM 2
If A is an event associated with the sample space S of a random experiment, then
P{S/A)=P{A/A)=1
PROUl-
We have.
PiSrsA) _P{A)
P(A)
"P(A)
Hence,
P (S/A) =P (A/A)=l
P(AnA)
P(A)
~T(Ar~~P(A)
P(S/A) =
= 1. Also, P (A/A) =
= 1
Q.E.D.
Read

Questions in LaTeX:

1. $P(A \cap B) = P(A)P(B|A)$
2. $P(A \cap B) = P(B)P(A|B)$
3. $0 < P(A|B) < 1$
4. $P(S|A) = P(A|A) = 1$

Extract only the math questions from the following text.
Output the questions strictly in LaTeX math notation format.
Do NOT add any explanations or other text.

Text:
MATHEMATICS-XIl
30.24
THEOREM 3
Let A and 6 be two events associated with a random experiment and S be the sample space.
If C is an event such that P (C) * 0, then
p({A u B)/C 1 = P (A/C) + P (B/C) - P {(A n B)/C)
In particular, if A and B are mutually exclusive events, then
= P(A/C) + P(B/C)
/
P
{A uB)/C
PrCXDF
We have.
P{(AuB)nC}
P
(A uB)/C
P(C)
P {{Ar^O^iBr^C)}
uB)/C
P
(A
P(C)
P(AnC)+ (BnC)-P{AnC)n(BnC)}
uB)/C
P
(A
P(C)
P (A n C) + P (B n C) - P (A n B n C)
uB)/C
P
(A
P(C)
P(AnC)
P(BnC)
P{AnBnC)
P
(A uB)/C
P(C)
P(C)
P(C)
P{A/C) + P{B/C)-P
(A nB)/C
P
(A uB)/C
If A and B are mutually exclusive events, then P
(A nB)/C
=0
P
(A nB)/C)
=P{A/C) + P{B/C)
Q.E.D.
THEOREM 4 ^A iind Bare two events associated with a random experiment, then P{A/B) =1 -P(A/B).
PROOF
We know that
P(S/B) =1
P
(A uA)/B
=1
[See Theorem 2]
[●.●A and A are mutually exclusive events]
Q.E.D.
P(A/B) + P(A/B)=1
P(A/B)=1-P(A/B)
Following examples will illustrate the applications of the above formulae and properties for
conditional probability.
ILLUSTRATIVE EXAMPLES
BASED ON BASIC CONCEPTS (BASIC)
EXAMPLE 1
If A and B are two events such that P(A) = 05, P (B) = 0.6 and P (A u B) = 0.8, find
P(A/B) andP{B/A).
SOLUTION
We have, P (A) = 05, P (B) = 0.6 and P (A u B) = 0.8
We know that
P (A u B) ^ P (A)+ P{B)-P {An B)
P(AnB) = P (A) + P (B) - P (A u B) = 05 + 0.6 - 0.8 = 0.3
Read

Questions in LaTeX:

1. Let $A$ and $B$ be two events associated with a random experiment and $S$ be the sample space. If $C$ is an event such that $P(C) = 0$, then
\[P\left(\frac{A \cup B}{C}\right) = P\left(\frac{A}{C}\right) + P\left(\frac{B}{C}\right) - P\left(\frac{A \cap B}{C}\right)\]
In particular, if $A$ and $B$ are mutually exclusive events, then
\[P\left(\frac{A \cup B}{C}\right) = P\left(\frac{A}{C}\right) + P\left(\frac{B}{C}\right)\]

2. We have
\[P\left(\frac{A \cap B}{C}\right) = P\left(\frac{A}{C}\right) \cdot P(C) + P\left(\frac{B}{C}\right) \cdot P(C) - P\left(\frac{A \cap B}{C}\right) \cdot P(C)\]
\[P\left(\frac{A \cup B}{C}\right) = P\left(\frac{A}{C}\right) \cdot P(C) + P\left(\frac{B}{C}\right) \cdot P(C) - P\left(\frac{A \cap B}{C}\right) \cdot P(C)\]
\[P\left(\frac{A \cup B}{C}\right) = P\left(\frac{A}{C}\right) + P\left(\frac{B}{C}\right) - P\left(\frac{A \cap B}{C}\right)\]

3. If $A$ and $B$ are mutually exclusive events, then
\[P\left(\frac{A \cap B}{C}\right) = 0\]
\[P\left(\frac{A \cup B}{C}\right) = P\left(\frac{A}{C}\right) + P\left(\frac{B}{C}\right)\]

4. If $A$ and $B$ are two events such that $P(A) = 0.5$,

Extract only the math questions from the following text.
Output the questions strictly in LaTeX math notation format.
Do NOT add any explanations or other text.

Text:
PROBABILITY
30.25
P(AnB)
0.3
1
PjAnB) ^ ^ _ 3
P{A)
" 05 " 5
P{A/B) =
= j and, P (B/A) =
P{B)
0.6
EXAMPLE 2
If A and B are two events such that P{A) = 0.3, P{B) = 0.6 and P (B/A) = 0.5, find
PiA/B)andP(AuB).
SOLUTION
We have, P (A) = 0.3, P (B) = 0.6 and P (B/A) = 0.5
P (A nB)
0.15 ^ 1
0.6
” 4
P(AnB)
P(A) P (B/A) = 0.3 x 0.5 = 0.15 and, P (A/B)
=
P(B)
Thus, we obtain: P(A) = 0.3, P(B) = 0.6 and P(AnB)
=
0.15
P(A^B)=P(A) + P(B)-P(AnB) =0.3+0.6-0.15=0.75
EXAMPLE 3
IfP (not A) = 07, P (B) = 07 and P^/A) = 0.5, then find P (A/B) and P(A'u B).
SOLUTION
We have, P (not A) = 07
or, P (A) = 07 => 1 - P (A) = 07 => P (A) = 0.3.
P(AnB)
P(AnB)
Now,
P(B/A)
=
=>
0.5
=
P (A r^B)
= 0.15
P(A)
P(AnB)
0.15
3
0.3
P(A/B)
=
P(B)
0.7
14
and,
EXAMPLE 4
If A and B are two events associated with a random experiment such thatP (A) =0.8,
P(B)=0.5,P(B/A) = 0A,find
(i)P(AnB)
(h)P(A/B)
(iii)P(AuB)
P(A uB) = P(A)^^^ P(B)-P(AnB) = 0.3 + 0.7 -0.15 = 0.85
SOLUTION
(i) We have,
P (B/A) =0.4
P(BnA)
P(BnA)
= 0.4
V P(B/A) =
P(A)
P(A)
P(AnB) = 0.4 => P(AnB)^ 0.32
[v P (A) = 0.8 (given)]
0.8
P (A n B)
(ii) We know that: P (A/B) =
P(B)
0.32
P(A/B) =
= 0.64
[●.●P(AnB) = 0.32 and P (B) = 05]
05
(iii) We know that:
P (A u B) = P (A) + P (B) - P (A n B)
P (A u B) = 0.8 + 05 - 0.32 = 0.98
EXAMPLES
Afair die is rolled. Consider the events A = {1, 3,5}, B = {2, 3} andC = {2, 3, 4,5}. Fmrf
[●.●P (A) = 0.8, P (B) = 05 and P (A r. B) = 0.32]
(i) P (A/B) and P (B/A)
(ii) P (A/C) and P (C/A)
(iii) P (A u B/C) and P (A n B/C)
SOLUTION
We have, S = {1, 2, 3, 4,5, 6}, A = {1, 3,5}, B = {2, 3} and C = {2, 3, 4, 5}
n(S) = 6,n(A) = 3,n(B) = 2 and n(C) = 4
P(A)=|=1 P(B)=^ = i,P(C)=^ = ^,P(AnB)=i, P(AnC)=|
^
6
2
6
3
6
3
6
6
P(BnC) =- = -,P (A nBnC)=i and P (Aw B)=-=-
6
3
6
6
3
P(A/S) =
3'
P(AnB)
1/6
1
P(B nA)
1/6 _ 1
P(A)
“l/2~3
P (A nC)
1/3 _ 2
P(A)
"1/2" 3
(i)
= — and, P (B/A) =
P(B)
1/3
2
P(AnC)
1/3
1
(ii)
P(A/C) =
= - and, P(C/A) =
P(C)
2/3
2
P
(A nB)nC
P(AnBnC) _ 1/6 _1
2/3 ”4
(iii)
P (A n B/C) =
P(C)
P(C)
Read

Questions in LaTeX:

\begin{enumerate}
\item{P(AnB)}
\item{P(A/B)}
\item{P(AuB)}
\end{enumerate}

Extract only the math questions from the following text.
Output the questions strictly in LaTeX math notation format.
Do NOT add any explanations or other text.

Text:
MATHEMATICS-XII
30.26
P(BnC)
1/3
1
P(C)
“2/3“ 2
P{AuB/C) = P(A/C) + (B/C)-P(AnB/C) == \ +
^
2
1
X
EXAMPLE6
Three events A, B and C have
probabilities
and - respectively. Given
that
5
3
2
P {/I nC) = ^andP{B nC) = -^,/mrf the values ofP (C/B)and P {A nC). INCERT EXEMPLAR]
SOLUTION
We have,
P(A)=-,P(B)=-,P(C)=i,P(AnC)=^ and P(BnC)=|
5
3
2
5
4
P(CoB)
1/4
3
P(B)
“1/3 “4
P(AnC)=P(AUC)=l- {P(A) + P(C)-P(AnC)} = l-(^| + i-~j =^
EXAMPLE?
// P (A)
^ {B)=^and P (A n B)=^,find P (A/B)and P (B/A).
P(AnB)
P(B/C) =
and.
4
1
P(C/B) =
and,
P(A nB)
and P(B/A)
=
SOLUTION
We know that
P{A/B)
=
P(A)
P(B)
Therefore, to find P (A/B) and P (6/A), we need the values of P (A nB), P (A) and P (B). So,
let us first, compute these probabilities.
Now,
P(A nB) = P(A wB)
= l-P(AuB) = l-{P(A) + P(B)-P(AnB)} =
+
P{A) = l-P(A) = ^ and P (B)
P(A/B) =
3
8
1
= l^P(B) = -
- and P(B/A)
=
4
8
PjAnB) _ ^ ^ 3
P (A
)
“ 5/8 " 5
P (A n B)
3/8
1/2
P(B
)
A die is rolled twice and the sum of the numbers appearing on them is observed to be 7. What
EXAMPLES
is the conditional probability that the number 2 has appeared at least once?
SOLUTION
Consider the following events:
A = Getting number 2 at least once; B = Getting 7 as the sum of the numbers on two dice.
We have.
A = {(2,1), (2, 2), (2, 3), (2, 4), (2,5), (2, 6), (1, 2), (3, 2), (4, 2), (5, 2) (6, 2)}
B = {(2,5), (5, 2), (6,1), (1,6), (3, 4), (4 3)}
and.
When a die is rolled twice, there are 36 elementary events.
P(B) = A and P(A nB) = ^
11
B(A) = 36'
36
36
P{AnB)
2/36
_
1
P(B)
6/36 “ 3
Required probability = P(A/B) =
So,
A black and a red die are rolled.
EXAMPLE 9
(i)
Find the conditional probability of obtaining a sum greater than 9, given that the black die resulted
hi a 5.
(ii)
Find the conditional probability of obtaining the sum 8, given that the red die resulted in a number
less than 4.
[CBSE 20181
Read

Questions in LaTeX:

\[
\begin{array}{l}
P(B) = \frac{1}{3} \\
P(C) = \frac{2}{3} \\
P(A \cap B) = 1 \\
P(A \cap C) = \frac{1}{3} \\
P(B \cap C) = \frac{1}{2} \\
\end{array}
\]

\[
\begin{array}{l}
P(A \cap B) = P(A) - P(A \cap C) \\
P(A \cap B) = 1 - \frac{1}{3} = \frac{2}{3} \\
P(A \cap B) = \frac{2}{3} \\
\end{array}
\]

\[
\begin{array}{l}
P(A \cap B) = P(A) + P(B) - P(A \cap B) \\
P(A \cap B) = \frac{2}{3} + \frac{1}{3} - \frac{1}{2} = \frac{1}{6} \\
P(A \cap B) = \frac{1}{6} \\
\end{array}
\]

\[
\begin{array}{l}
P(A \cap B) = P(A) - P(A \cap C) \\
P(A \cap B) = \frac{2}{3} - \frac{1}{3} = \frac{1}{3} \\
P(A \cap B) = \frac{1}{3} \\
\end{array}
\]

\[
\begin{array}{l}
P(A \cap B) = P(A) + P(B) - P(A \cap B) \\
P(A \cap B) = \frac{2}{3} + \frac{1}{3} - \frac{1}{2} = \frac{1}{6} \\
P(A \cap B) = \frac{1}{6} \\
\end{array}
\]

\[
\begin{array}{l}
P(A \cap B) = P(A) - P(A

Extract only the math questions from the following text.
Output the questions strictly in LaTeX math notation format.
Do NOT add any explanations or other text.

Text:
PROBABILITY
30.27
Consider the following events:
A = Getting a sum greater than 9, B = Getting 5 on black die
C = Getting 8 as the sum, D = Getting a number less than 4 on red die.
SOLUTION
Clearly,
A = {(4, 6), (6, 4), (5,5), (6,5), (5, 6), (6, 6)}, B = {(5,1), (5, 2), (5, 3), (5, 4), (5,5), (5, 6)}
C = {(2,6),(6, 2), (4, 4), (3,5), (5, 3)}
D = {(1,1), (2,1), {3,1), (4,1), (5,1), (6,1), (1, 2), (2, 2), (3, 2), (4, 2), (5, 2), (6, 2),
(1, 3), (2, 3), (3, 3), {4, 3), (5, 3), {6, 3)}
Clearly, n{A) = 6, n{B) = 6, «(C) = 5, n{D) = 18, n (A n B) = 2 and h (C n D) = 2.
and, P(CnD)=—=
36
18
and.
18
1
2
1
-,P(AnB)=:~ = —
-
36
18
36
2'
1
P(AnB) _ 1/18 _ 1
P(B)
~J/6~3
P(CnD)_l/18_l
P(D)
“ 1/2 ~9
(i)
Required probability =P(A/B) =
(ii)
Required probability = P (C/D) =
BASED ON LOWER ORDER THINKING SKILLS (LOTS)
EXAMPLE 10
Two integers are selected at random from integers 1 through 11. If the sum is even,find the
probability that both the numbers are odd.
SOLUTION
Out of integers from 1 to 11, there are 5 even integers and 6 odd integers.
Consider the following events:
A = Both the numbers chosen are odd
, B = The sum of the numbers chosen is even
Since the sum of two integers is even if either both are even or both are odd.
6^^2
6/~-C2
^C2-t^C2
P(A) =
. P(B)
=
and P (A r\B)
=
PjAnB) _^C2/^^C2 ^
^C2
C2+%
+ %
15 + 10
5
11
11
11
C2
C2
C2
15
3
Required probability = P (A/B) =
6
P(B)
11 C2
10% of the bulbs produced in afactory are red colour and 2% are red and defective. If one
EXAMPLE 11
bulb is picked at random, determine the probability of its being defective if it is red.
SOLUTION
Consider the following events:
A = The bulb produced is red, B = The bulb produced is defective.
1
2
1
= —and P{AnB) =—= —
100
50
P(AnB)
1/50
1
[NCERT EXEMPLAR]
10
It is given that P(A) = 100
10
Required probability = P(B/A) =
P{A)
1/20
5
EXAMPLE 12
A couple has 2 children. Find the probability that both are boys, if it is known that (i) one of
the children is a boy (ii) the older child is a boy.
SOLUTION
Let B,- and Gj stand for
child be a boy and girl respectively. Then the sample
space can be expressed as S = {Bj B2, B| G2,
B2, Gj G2}.
Consider the following events:
A = Both the children are boys; B = One of the children is a boy; C = The older child is a boy.
ICBSE 2010,2014]
Read

Questions in LaTeX:

\begin{enumerate}
\item{Consider the following events: A = Getting a sum greater than 9, B = Getting 5 on black die. C = Getting 8 as the sum, D = Getting a number less than 4 on red die. Clearly, n{A) = 6, n{B) = 6, «(C) = 5, n{D) = 18, n (A n B) = 2 and h (C n D) = 2. and, P(CnD)=—=
\end{enumerate}

\begin{enumerate}
\item{Two integers are selected at random from integers 1 through 11. If the sum is even, find the probability that both the numbers are odd.
\end{enumerate}

\begin{enumerate}
\item{Two bulbs are selected at random from a batch of 100 bulbs, 10 of which are defective. If one bulb is found to be defective, find the probability that the other bulb is also defective.
\end{enumerate}

\begin{enumerate}
\item{A couple has 2 children. Find the probability that both are boys, if it is known that (i) one of the children is a boy (ii) the older child is a boy.
\end{enumerate}

\begin{enumerate}
\item{A couple has 2 children. Find the probability that both are boys, if it is known that (i) one of the children is a boy (ii) the older child is a boy.
\end{enumerate}

\begin{enumerate}
\item{A couple has 2 children. Find the probability that both are boys, if it is known that (i) one of the children is a boy (ii) the older child is a boy.
\end{enumerate}

\begin{enumerate}
\item{A couple has 2 children. Find the probability that both are boys, if it is known that (i) one of the children is a boy (ii) the older child is a boy.
\end{enumerate}

\begin{enumerate}
\item{A couple has 2 children. Find the probability that both are boys, if it is known that (i) one of the children

Extract only the math questions from the following text.
Output the questions strictly in LaTeX math notation format.
Do NOT add any explanations or other text.

Text:
MATHEMATICS-XII
30.28
A = {B| 82)/ B = {Bj G2/ Bj B2, Gj B2I and C — {Bj 82, B| G2)
A nB = {Bj 82} and AnC = {Bj 83}
Required probability = P{A/B) =
Then,
P (A nB)
_ 1/4 _
1
(i)
3/4
3
P{AnC)
_ 1/4 _
1
2/4
2
P(B)
Required probability = P(A/C) =
(ii)
P(C)
Consider a random experiment in which a coin is tossed and if the coin shows head it is
EXAMPLE 13
tossed again but if it shows a tail then a die is tossed. If 8 possible outcomes are equally likely, find the
probability that the die shozos a number greater than 4 if it is knoivn that the first throzv of the coin results
in a tail.
[CBSE2014]
SOLUTION
The sample space S associated with the given random experiment is
S = {{H,H), (H,T), (T,1),(7',2),{T,3)(T,4), (T, 5), (T, 6)}.
Let A be the event that the die shows a number greater than 4 and B be the event that the fiist
throw of the coin results in a tail. Then,
A = {(T,5)(T,6)} and B = {(T, 1), (T, 2), (T, 3), (T, 4), (T, 5), (T, 6)}
P{AnB) _ njAnB) _ 2 ^
1_
6 "
3
Required probability = P(A/B) =
n{B)
P{B)
EXAMPLE 14
A coin is tossed twice and the four possible outcomes are assumed to be equally likely. IfA
is the event, ‘both head and tail have appeared', and B be the event, ‘at most one tail is observed', find
P (A), P (B), P (A/B) and P (B/A).
SOLUTION
Here,S = {HH,HT,TH,rT},A = {HT,rH} and B = {HH,HT,TH}.
AnB = {HTJH}.
n{A) ^2^1
n{S)
4
P(AnB)
1/2
n (A n B)
_ 2 _ 1
n(S)
4'2
»(B)
- and, P (A n B) =
4
P(AnB)
^,P(B) =
P{A) =
Now,
«(S)
1/2
- and P(B/A) =
3
=
1.
P(A/B) =
1/2
P{A)
3/4
P(B)
EXAMPLE 15
A bag contains 3 red and 4 black balls and another bag has 4 red and 2 black balls. One bag
is selected at random and from the selected bag a ball is draivn. Let A be the event that the first bag is
selected, B be the event that the second bag is selected and C be the event that the ball drawn is red. Find
P(A), P(B), P (C/A) and P (C/B).
1
1
SOLUTION
There are two bags. Therefore, P (A) = - and P (B) = -
P(C/A) = Probability of drawing a red ball when first is selected
3
= Probability of drawing a red ball from first bag
= —
4
P (C/B) = Probability of drawing a red ball from second bag = —
Now,
2
and.
3
EXAMPLE 16
A coin is tossed, then a die is thrown. Find the probability of obtaining a ‘6‘given that head
came up.
SOLUTION
The sample space S associated to the given random experiment is given by
S = {(H, 1), (H, 2), (H, 3), (H, 4), (H, 5), (H, 6), (T, 1), (T, 2), (T, 3), (T, 4), (T, 5), (T, 6)}
Consider the events: A = Getting head on the coin.
Clearly, A = {(H, 1), (H, 2), (H, 3), (H, 4), (H, 5), (H, 6)}and B = {{H, 6), (T, 6)}
P (A n B)
1/12 ^ 1
6/12
6
B = Getting 6 on the dice.
Required probability = P(B/A) =
P{A)
Read

Questions in LaTeX:

\begin{enumerate}
\item A = \{B \mid 82 \} \quad B = \{B_j G_2, B_j B_2, G_j B_2, C \setminus \{B_j 82, B \mid G_2\} \} \quad A \cap B = \{B_j 82\} \quad A \cap C = \{B_j 83\} \quad P(A \cap B) = \frac{1}{4} \quad P(A \cap C) = \frac{1}{4} \quad P(B) = 1 \quad P(A \mid B) = \frac{P(A \cap B)}{P(B)} = \frac{1/4}{1/4} = 1 \quad P(A \mid C) = \frac{P(A \cap C)}{P(C)} = \frac{1/4}{1/4} = 1
\item P(A \cap B) = \frac{1}{4} \quad P(A \cap C) = \frac{1}{4} \quad P(B) = 1 \quad P(A \mid B) = \frac{P(A \cap B)}{P(B)} = \frac{1/4}{1/4} = 1 \quad P(A \mid C) = \frac{P(A \cap C)}{P(C)} = \frac{1/4}{1/4} = 1
\item P(A \cap B) = \frac{2}{4} \quad P(A \cap C) = \frac{1}{4} \quad P(B) = \frac{1}{4} \quad P(A \mid B) = \frac{P(A \cap B)}{P(B)} = \frac{2/4}{1/4} = 2 \quad P(A \mid C) = \frac{P(A \cap C)}{P(C)} = \frac{1/4}{1/4} = 1
\end{enumerate}

EXAMPLE 13:

\begin{enumerate}
\item The sample space S associated with the given random experiment is S = \{(H,H), (

Extract only the math questions from the following text.
Output the questions strictly in LaTeX math notation format.
Do NOT add any explanations or other text.

Text:
30.29
PROBABILITY
A committee of 4 students is selected at random from a group consisting of 8 bops and 4
EXAMPLE 17
girls. Given that there is at least one girl in the committee, calculate the probabilitp that there are exactly 2
girls in the committee.
SOLUTION
Consider the following events:
A - There is at least one girl on the committee, B = There are exactly 2 girls on the committee.
P{AnB)
(NCERT EXEMPLAR)
We have to find P{B/A).'We know that: P{B/A) =
P(A)
Now,
70 _85
495 ” 99
P(A)=1-?(A)=1-
= 1-
12 C4
^C2X ^C4 _6x28
56
P{A n 6) = P( Selecting 2 girls and 2 boys out 8 boys and 4 girls =
12
495
165
C4
P(AnB)
56
. 85 _ 168
P{A)
“T^
■99 “425
P{B/A) =
Two coins are tossed. What is the probability of coming up two heads if it is knozvn that at
EXAMPLE 18
least one head comes up.
Consider the events: A = Getting at least one head, B = Getting two heads.
SOLUTION
Clearly, A = {HT, TH, HH}, B = {HH} and so A n B = |H H)
P (A) = ~,P {An B) =
1
[v
S = {HH,HT,TH,TT\]
4
P{AnB)
_
_
l_
P(A)
“ 's/a “ 3
Required probability = P(B/A) =
An instructor has a test bank consisting of 300 easy True/False questions, 200 difficult
EXAMPLE 19
True/False questions, 500 easy multiple choice questions (MCQ) and 400 difficult midtiple choice
questions. If a questions is selected at random from the test bank, zohat is the probability that it will be an
easy question given that it is a multiple choice question.
SOLUTION
Consider the following events;
£ = The question selected is an easy question,
D = The question selected is a difficult question
T = The question selected is a True/False question,
M = The question selected is a multiple choice question.
Total number of questions = 300 + 200 + 500 + 400 =1400
500 _ 5
1400 ~14'
600 = ~,P(T) =
800
|,P(D) =
P(E) =
1400
7
'
- and P(EnM) =
1400
7
'
500 _ 5
1400 “ 14
P(EnM)
_ 5/14 _ 5
P(M)
“ 9/14 “9
900
P(M) = 1400
14
Required probability = P (£/M) =
EXAMPLE 20
A die is thrown three times. Events A and B are defined as follows:
A:4 on the third throw, B:6 on the first and 5 on the second throzo.
Find the probability of A given timt B has already occurred.
SOLUTION
There
are 6x6x6= 216
elementary
events
associated
with
the
random
experiment.
Read

Questions in LaTeX:

\[
\text{What is the probability that there are exactly 2 girls in the committee given that there is at least one girl in the committee?}
\]

\[
\text{What is the probability of coming up two heads if it is known that at least one head comes up when two coins are tossed?}
\]

\[
\text{What is the probability that a question selected at random from the test bank will be an easy question given that it is a multiple choice question?}
\]

\[
\text{What is the probability of event A given that event B has already occurred when a die is thrown three times?}
\]

Extract only the math questions from the following text.
Output the questions strictly in LaTeX math notation format.
Do NOT add any explanations or other text.

Text:
30.30
MATHEMATICS-XII
Clearly,
A = {(1,1, 4), (1, 2, 4), (1, 3, 4), (1, 4, 4), (1,5, 4), (1, 6, 4),(2,1, 4), (2, 2, 4), (2, 3, 4),
(2, 4, 4), (2,5, 4), (2, 6, 4),(3,1, 4), (3, 2, 4), (3, 3, 4), (3, 4, 4), (3,5, 4), (3, 6, 4)
(4,1, 4), (4, 2, 4), (4, 3, 4), (4, 4, 4), (4, 5, 4), (4, 6, 4), (5,1, 4), (5, 2, 4), (5, 3, 4),
(5, 4, 4), (5,5, 4), (5, 6, 4), (6,1, 4), (6, 2, 4), (6, 3, 4), (6, 4, 4), (6,5, 4), (6, 6, 4)}
B = {(6,5,1), (6,5, 2), (6,5, 3), (6,5, 4), (6,5,5), (6,5, 6)1
AnB = {(6,5, 4)}.
and.
We observe that n (A) = 36, n(B)=6 and n(A n B) =1
P(AnB)=— and P(B)= —
216
216
P(AnB)
_ 1/216 _ 1
P(B)
" 6/216 ~6
Required probability =
Three dice are thrown at the same time. Find the probability of getting three two's if it is
[NCERT EXEMPLAR]
EXAMPLE 21
known that the sum of the numbers on the dice was a six.
Associated
to
the random
experiment
of
throwing
three
dice
there
are
SOLUTION
6x6x6= 216 elementary events.
Consider the following events: A =Sum of the numbers on the dice is six, B =Getting three twos
We have to find P(B/A). We observe that
A ={(1, 2, 3), (1, 3, 2), (2, 3,1), (2,1, 3), (3,1, 2), (3, 2,1), (1,1, 4), (1, 4,1), (4,1,1), (2, 2, 2)1
andB = {(2, 2, 2))
10
P(B) =—andP(AnB)= —
P(A) = 216'
216
216
PjAr^B) _ 1/216 _ 1
P(A)
10/216 "io
Hence,
Required probability = P(B/A) =
In a hostel 60% of the students read Hindi newspaper, 40% read English newspaper and
EXAMPLE 22
20 % read both Hindi and English newspapers. A student is selected at random.
(i)
Find the probability that she reads neither Hindi nor English news papers.
(ii)
If she reads Hindi newspaper, find the probability that she reads English newspaper.
(iii)
If she reads English newspaper, find the probability that she reads Hindi neiuspaper.
SOLUTION
Consider the following events:
H = Student reads Hindi newspaper, £ = Student reads English newspaper.
20
1
60
40
- and P(Hn£) =
We find that: P (H) = 100
5
'
100
5
Required probability = P(Hn£)= P {H <jE)= l-P(HuE)
100
5
(i)
= l-{P(H) + P(£)-P(HnE)} = 1-jl + l-l
[a
D
0
4
1
= 1 --
5
5
P(Hn£)
1/5
1
P (H)
“ ^ “ 3
P(Hn£) _l/5 _1
P(£)
“X^"2
EXAMPLE 23
electronic assembly consists of two subsystems say A and B. From previous testing
procedures, thefolloioing probabilities are assumed to be known.
P (A fails) = 0.2, P (B fails alone) = 0.15, P (A and Bfail) = 0.15.
Evaluate the following probabilities:
(i)
P(Afails/B has failed)
(ii) P (A fails alone)
(ii)
Required probability
= P (E/H) =
(iii)
Required probability = P (H/£) =
Read

Questions in LaTeX:

\begin{enumerate}
\item Extract only the math questions from the following text.
\item Output the questions strictly in LaTeX math notation format.
\end{enumerate}

Answers:

\begin{enumerate}
\item
\begin{align*}
P(A \cap B) &= \frac{1}{216}, \\
P(A) &= \frac{216}{216}, \\
P(B) &= \frac{6}{216}, \\
P(B|A) &= \frac{1}{216}.
\end{align*}
\item
\begin{align*}
P(B|A) &= \frac{P(B \cap A)}{P(A)}, \\
P(B|A) &= \frac{1/216}{216/216}, \\
P(B|A) &= \frac{1}{216}.
\end{align*}
\item
\begin{align*}
P(H|E) &= \frac{P(H \cap E)}{P(E)}, \\
P(H|E) &= \frac{3/216}{6/216}, \\
P(H|E) &= \frac{1}{2}.
\end{align*}
\item
\begin{align*}
P(E|H) &= \frac{P(E \cap H)}{P(H)}, \\
P(E|H) &= \frac{1/216}{10/216}, \\
P(E|H) &= \frac{1}{10}.
\end{align*}
\end{enumerate}

Extract only the math questions from the following text.
Output the questions strictly in LaTeX math notation format.
Do NOT add any explanations or other text.

Text:
30.31
PROBABILITY
SOLUTION
Consider the following events: E=A fails, F = B fails. It is given that
P {A fails) = 0.2, P {A and B fails) = Ol^and, P (B fails alone) = 0.15
P(£) = 0.2, P(£nP) = 0.15 and P(£ nP) = 0.15
Now,
P (£ n F) = 0.15
P (F) - P (£ o F) = 0.15
[v P(£nF)= 0.15]
P (F) = P (£ n F) + 0.15 = 0.15 + 0.15 = 0.30
P (£ n F)
0.15
1
P (A fails/B has failed) = P(£/F) =
(i)
0.30
2
P (/I fails alone) = P (£ nF) = P {£)-P (£ n F) =0.2 -0.15 = 0.05
P(F)
(ii)
BASED ON HIGHER ORDER THINKING SKILLS (HOTS)
Three distinguishable balls are distributed in three cells. Find the conditional probabilih/
that all the three occupy the same cell, given that at least txvo ofthexn are in the same cell.
SOLUTION
Since each ball can be placed in a cell in three ways. Therefore, three distinct balls
can be placed in three cells in 3 x 3 x 3 = 27 ways.
Consider the following events:
£ = All balls are in the same cell, F = At least two balls are in the same cell.
EXAMPLE 24
All balls can be placed in the same cell in three ways.
27
Now, P (F) = P (At least two balls are in the same cell) =1 - P (Balls are placed in distinct cells)
^
P(F)=1-
3!
21
27
27
Clearly, £cF=>£nF=£=> P(£oF)=P(£)= —
Required probability = P (£/F) =
27
P(£nF) _ 3/27 _1
P(F)
"21/27 "7
Consider the experiment of tossing a coin. If the coin shows head toss it again but if it
EXAMPLE 25
shows tail then throw a die. Find the conditional probability of the event 'the die shows a number greater
than 4, given that 'there is at least one tail'.
[NCERT]
The outcomes of the experiment can be represented in the following tree diagram.
SOLUTION
(HH)
Head(
(RT)
(T5)
(T,6)
Fig. 30.3 Outcomes of the random experiment
The sample space S of the experiment is given as
S = {(H, H), (H, 7), (T, 1), (T, 2), (T, 3), (T, 4), (T, 5), (T, 6)}
The probabilities of these elementary events are:
P{(H,H)}=ixi=i,P{(H,T)} = lxl=i,Pi(T,l)} = lxi = L.
1
1
1
1
1
1
-
, P {(T, 3)} =- X i = — , P {(T, 4} = - X - =12,
6
12
2
6
12
2
6
1
1
1
P{(T, 2}=-
Read

Questions in LaTeX:

1. $P(A \text{ fails}) = 0.2$
2. $P(A \text{ and } B \text{ fails}) = 0.15$
3. $P(B \text{ fails alone}) = 0.15$
4. $P(F) = P(A \text{ fails} \text{ and } B \text{ fails}) + P(B \text{ fails alone}) = 0.15 + 0.15 = 0.30$
5. $P(A \text{ fails} \text{ and } B \text{ fails}) = P(F) - P(A \text{ fails} \text{ and } B \text{ fails alone}) = 0.30 - 0.15 = 0.15$
6. $P(A \text{ fails alone}) = P(A \text{ fails} \text{ and } B \text{ fails}) - P(A \text{ fails} \text{ and } B \text{ fails alone}) = 0.15 - 0.15 = 0.05$
7. $P(F) = P(A \text{ fails} \text{ and } B \text{ fails}) + P(B \text{ fails alone}) = 0.15 + 0.15 = 0.30$
8. $P(A \text{ fails} \text{ and } B \text{ fails alone}) = P(A \text{ fails}) - P(F) = 0.2 - 0.30 = -0.10$
9. $P(F) = P(A \text{ fails} \text{ and } B \text{ fails}) + P(B \text{ fails alone}) = 0.15 + 0.15 = 0.30$
10. $P(A \text{ fails} \text{ and } B \text{ fails alone}) = P(A \text{ fails}) - P(F) = 0.2 - 0.30 = -0.10$
11. $P(F) = P(A \text{ fails} \text{ and }

Extract only the math questions from the following text.
Output the questions strictly in LaTeX math notation format.
Do NOT add any explanations or other text.

Text:
30.32
MATHEMATtCS-Xli
P{(T,5)}=-xi=J-and,P{(r,6)} = -xl= —
2
6
12
2
6
12
Consider the following events:
A = The die shows a number greater than 4, B = There is at least one tail.
Clearly, A = {(T, 5), (T, 6)}, B = {(H, T), (T, 1), (T, 2), (T, 3), (T, 4), (T, 5), (T, 6)}
and,
AnB = {(T,5)(T,6)}
P (B) = P {(H, T)} 4- P {T, 1)} + P {(T, 2)} + P {(T, 3)} + P {(T, 4)} + P {(T, 5)} + P {(T, 6)}
1
1
1
1
1
1
1
3
4
12
12
12
12
12
12
4
P (A n B) = F {(T, 5)} + P {(T, 6)} =:! +:! =
12
12
o
1
[See Fig. 30.4]
1
and.
1/4
(H,H)
Head (H)
(H,T)
(Tl)
(Z2)
(T,3)
(T,4)
(Z5)
(T,6)
Fig, 30,4 Computation of probabilities of the outcomes of the experiement
P(AnB) ^ 1/6 _ 4 _2
P(B)
3/4
18
9
REMARK
Here, the elementary events are not equally likely. So, we cannot say that
P(B)=-,P(AnB)=-andso P(A/B)=^-^^^-^=— = -
8
^
^8
P(B)
7/B
7
EXAMPLE 26
Consider the experiment of throwing a die, if a multiple of 3 comes up throw the die again
and if any other number comes toss a coin. Find the conditional probability of the event 'the coin shows a
tail', given that 'at least one die shows a 2'.
SOLUTION
The sample space of the experiment is given by
S = {(3,1), (3, 2) {3, 3), (3, 4),(3,5),(3,6),(6,1),(6, 2), (6, 3), (6, 4), {6,5), (6, 6)
(1. H), (1, T), (2, H), (2, T), (4, H), (4, T), (5, H), (5, T)}
The probabilities of the elementary evente are:
P {(3,l)} =-xi=—,P {(3,2)} = -xi=—,P{(3, 3)}=lxi= ^
6
6
36
6
6
36
'
r 36'
r ¥'"
r 36'
P{(6,4))=ixi=^,P((6,5)} = lxi = ^,P{(6,6)}=lxi= ^
D
D
ao
o
6
Jo
P {(l,H)} = -xi=—,P{(l,T)} = ~xi = —,P {(2,H)} =ix- =
6
2
12
^
6
2
12
^
6
2
12
1
1
1
1
1
1
1
1
1
P{(2,T)} = -x- = —,pp, H)} = ixi=—,P{(4,T)}=-xi=-
6
2
12
6
2
12
^
6
2
Required probability =P(2l/B) =
1
6
6
36'
1
1
1
6
6
36'
1
1
2
12
'
Read

Questions in LaTeX:

1. $A = \{(T,5), (T,6)\}$
2. $B = \{(H,T), (T,1), (T,2), (T,3), (T,4), (T,5), (T,6)\}$
3. $A \cap B = \{(T,5), (T,6)\}$
4. $P(B) = \frac{1}{4} + \frac{1}{6} + \frac{1}{6} + \frac{1}{6} + \frac{1}{6} + \frac{1}{6} + \frac{1}{6} = \frac{7}{6}$
5. $P(A \cap B) = \frac{1}{6} + \frac{1}{6} = \frac{2}{6}$
6. $P(A|B) = \frac{P(A \cap B)}{P(B)} = \frac{\frac{2}{6}}{\frac{7}{6}} = \frac{2}{7}$

7. $S = \{(3,1), (3,2), (3,3), (3,4), (3,5), (3,6), (6,1), (6,2), (6,3), (6,4), (6,5), (6,6)\}$
8. $P(A) = \frac{1}{6} + \frac{1}{6} = \frac{1}{3}$
9. $P(B) = \frac{1}{6} + \frac{1}{6} + \frac{1}{6} + \frac{1}{6} + \frac{1}{6} + \frac{1}{6} = \frac{1}{6}$
10. $P(A \cap B) = \frac{1}{6} + \frac{1}{6} = \frac{1}{3}$
11. $P(A|B) = \frac{P(A \cap B)}{P(B)} = \frac{\frac{1}{3}}{\frac{1}{6}} = \frac{2}{3}$

12. $S = \{(3,1), (3,2), (3,

Extract only the math questions from the following text.
Output the questions strictly in LaTeX math notation format.
Do NOT add any explanations or other text.

Text:
30.33
PROBABILITY
P {(5, H)} = -xi=—,P {(5,T)}=-x- =
”
6
2
12
6
1
1
2
12
Clearly, the elementary events are not equally likely.
Cor\sider the following events;
A = The coin shows a tail, B = At least one die shows a 2.
Clearly,
A = {(l,T),(2,D,(4,D,(5,r)},B = {(3,2),(6,2),(2,H),(2,T)} and, AnB = {(Z,T)}
P(B)=P{{3, 2)}+P{{6,2)} +
P(AnB)=P {(2,T)} = — = —
V
y
12
12
J_
J_
J_
J_^2
36
36
12
12 " 9
and.
1
P (A n B) _ 12 _ 9
3
2
24
8
Hence, Required probability =P(A/B) =
4
REMARK
As the elementary events are not equally likely. Therefore, we cannot say thatP (B) =
P(AnB)= — andsoP{A/B) = ^^^‘^^^ =1^ = 1.
^
^20
^
P (B)
4/20
4
P(B)
9
EXERCISE 30.3
BASIC
P (B) =— and P (A n B) =^ find P (A/B).
13
13
If A and B are events such that P (A) = 0.6, P (B) = 0.3 and P {A n B) = 0.2, find P (A/B) and
P{B/A).
If A and B are two events such that P (A n B) =0.32 and P (B) = 05, find P (A/B).
If P (A) = 0.4, P (B) = 0.8, P (B/A) = 0.6. Find P (A/B) and P (A u B).
If A and B are two events such that
(i) P (A) =1/3, P (B) =1/4 and P (A u B) =5/12, find P (A/B) and P (B/A).
(ii) P (A) =
^ (®) =
2nd P (A u B) =^, find P (A n B), P (A/B), P (B/A)
(iii) P (A) =4.(B) =
andP (A n B) =— , find P (A/B).
13
IfP(A)=-
1.
13'
2.
3.
4.
5.
13'
13
1
1
1
(iv)
P(A) =-, P(B) = - andP (A nB) = -, find P(A/B), P(B/A), P(A/B) and P(A/B).
[NCERT EXEMPLAR]
If A and B are two events such that 2 P (A) = P (B) =— and P (A/ B) = ^, find P (A u B).
13
3
If P (A) =^ , P (B) =^ and P (A u B) =
find
6.
7.
11
11
(iii) P(B/A)
(ii) P(A/B)
(i)P(AoB)
A coin is tossed three times. Find P (A/B) in each of the following;
8.
(i) A = Heads on third toss, B = Heads on first two tosses
(ii) A = At least two heads, B = At most two heads
(iii) A = At most two tails, B = At least one tail.
Two coins are tossed once. Find P (A/B) in each of the following:
(i) A = Tail appears on one coin, B = One coin shows head,
(ii) A = No tail appears, B = No head appears.
9.
Read

Questions in LaTeX:

\begin{align*}
P(\text{H}) &= \frac{1}{2}, \\
P(\text{T}) &= \frac{1}{2}, \\
P(\text{H}) &= \frac{1}{2}, \\
P(\text{T}) &= \frac{1}{2}, \\
P(\text{H}) &= \frac{1}{2}, \\
P(\text{T}) &= \frac{1}{2}, \\
P(\text{H}) &= \frac{1}{2}, \\
P(\text{T}) &= \frac{1}{2}, \\
P(\text{H}) &= \frac{1}{2}, \\
P(\text{T}) &= \frac{1}{2}.
\end{align*}

\begin{align*}
P(A) &= 0.6, \\
P(B) &= 0.3, \\
P(A \cap B) &= 0.2.
\end{align*}

\begin{align*}
P(A \cap B) &= 0.32, \\
P(B) &= 0.5.
\end{align*}

\begin{align*}
P(A) &= 0.4, \\
P(B) &= 0.8, \\
P(B|A) &= 0.6.
\end{align*}

\begin{align*}
P(A) &= \frac{1}{3}, \\
P(B) &= \frac{1}{4}, \\
P(A \cup B) &= \frac{5}{12}.
\end{align*}

\begin{align*}
P(A) &= 0.4, \\
P(B) &= 0.8, \\
P(A \cap B) &= 0.2.
\end{align*}

\begin{align*}
P(A) &= 0.5, \\
P(B) &= 0.5, \\
P(A \cap B) &= 0.
\end{align*}

\begin{align*}
P(A) &= 0.4, \\
P(B) &= 0.6, \\
P(A \cap B) &=

Extract only the math questions from the following text.
Output the questions strictly in LaTeX math notation format.
Do NOT add any explanations or other text.

Text:
30.34
MATHEMATICS-XII
A die is thrown three times. Find P (A/B) and P (B/A), if
A = 4 appears on the third toss, B = 6 and 5 appear respectively on first two tosses.
Mother, father and son line up at random for a family picture. If A and B are two evente
given by A = Son on one end, B = Father in the middle, find P (A/B) and P (B/A).
BASED ON LOTS
10.
11.
A dice is thrown twice and the sum of the numbers appearing is observed t'- be 6. What is
the conditional probability that the number 4 has appeared at least once?
Two dice are thrown. Find the probability' that the numbers appeared has the sum 8, if it is
known that the second die always exhibits 4.
A pair of dice is thrown. Find the probability of getting 7 as the sum, if it is known that the
second die always exhibits an odd number.
A pair of dice is thrown. Find the probability of getting 7 as the sum if it is known that the
second die always exhibits a prime number.
A die is rolled. If the outcome is an odd number, what is the probability that it is prime?
A pair of dice is thrown. Find the probability of getting the sum 8 or more, if 4 appears on
the first die.
Find the probability that the sum of the numbers showing on two dice is 8, given that at
least one die does not show five.
12.
13.
14.
15.
16.
17.
18.
Two numbers are selected at random from integers 1 through 9. If the sum is even, find the
probability that both the numbers are odd.
A die is thrown twice and the sum of the numbers appearing is observed to be 8. What is the
conditional probability that the number 5 has appeared at least once?
Two dice are thrown and it is known that the first die shows a 6. Find the probability that
the sum of the numbers showing on two dice is 7.
A pair of dice is thrown. Let £ be the event that the sum is greater than or equal to 10 and F
be the event "5 appears on the first-die”. Find P (£/F).If F is the event ”5 appears on at least
one die”, find P (£/f).
19.
20.
[CBSE2003]
21.
22.
The probability that a student selected at random from a class will pass in Mathematics is
4/5, and the probability that he/she passes in Mathematics and Computer Science is 1/2.
What is the probability that he/she will pass in Computer Science if it is known that he/she
has passed in Mathematics?
The probability that a certain person will buy a shirt is 0.2, the probability that he will buy a
trouser is 0.3, and the probability that he will buy a shirt given that he buys a trouser is 0.4.
Find the probability that he will buy both a shirt and a trouser. Find also the probability tliat
he will buy a trouser given that he buys a shirt.
In a school there are 1000 students, out of which 430 are girls. It is known that out of 430,
10% of the girls study in class XII. What is the probability that a student chosen randomly
studies in class XII given that the chosen student is a girl?
Ten cards numbered 1 through 10 are placed in a box, mixed up thoroughly and then one
card is drawn randomly. If it is known that the number on the drawn card is more than 3,
what is the probability that it is an even number?
Assume that each born child is equally likely to be a boy or a girl. If a family has two
children, what is the constitutional probability that both are girls? Given that
(i) the youngest is a girl
23.
24.
25.
26.
27.
(ii) at least one is girl.
(CBSE 20141
ANSWERS
4
2
1
3.
0.64
(iii) I
(iv)
4.
0.3,0.96
1.
^
3 ' 3
4
4
2
2
1
3
115
4'2'4'8
11
5.
(i)
(ii) -T
2-6
3'2
n
' 5
' 3
Read

Questions in LaTeX:

\begin{enumerate}
\item P(A) = \frac{1}{6}
\item P(B) = \frac{1}{6}
\item P(A|B) = \frac{1}{6}
\item P(B|A) = \frac{1}{6}
\end{enumerate}

\begin{enumerate}
\item P(A) = \frac{1}{3}
\item P(B) = \frac{1}{3}
\item P(A|B) = \frac{1}{3}
\item P(B|A) = \frac{1}{3}
\end{enumerate}

\begin{enumerate}
\item P(A) = \frac{1}{6}
\item P(B) = \frac{1}{6}
\item P(A|B) = \frac{1}{6}
\item P(B|A) = \frac{1}{6}
\end{enumerate}

\begin{enumerate}
\item P(A) = \frac{1}{6}
\item P(B) = \frac{1}{6}
\item P(A|B) = \frac{1}{6}
\item P(B|A) = \frac{1}{6}
\end{enumerate}

\begin{enumerate}
\item P(A) = \frac{1}{6}
\item P(B) = \frac{1}{6}
\item P(A|B) = \frac{1}{6}
\item P(B|A) = \frac{1}{6}
\end{enumerate}

\begin{enumerate}
\item P(A) = \frac{1}{6}
\item P(B) = \frac{1}{6}
\item P(A|B) = \frac{1}{6}
\item P(B|A) = \frac{1}{6}
\end{enumerate}

\begin{enumerate}
\item P(A) = \frac{1}{6}
\item P(B) = \frac{1}{6}
\item P(A|B) = \frac{

Extract only the math questions from the following text.
Output the questions strictly in LaTeX math notation format.
Do NOT add any explanations or other text.

Text:
30.35
PROBABILITY
1
J_
6' 36
8. (i) ^ (ii) I (iii) |
1
5
1
25. —
(ii) I (iii) I
2
12. T
..
4
9.
(i)
1
(ii)
0
7.
(i)
10. T
11
2
1
1
1
11. 1^2
1
17. T
16. -
14. T
15. T
6
3
6
5
1
_3
3'11
2
1
3
22. -
(iii) I
20.
r
18.
5
25
2
1
4
5
27.
(i) -
HINTS TO SELECTED PROBLEMS
24. 0.12, 0.6
26. -
23.
10
7
8
6.
P{AnB)=P{B)P(A/B)=^x^=^
P{AuB) = P{A) + P{B)-P{AnB)=^ + ^-^ =^
26
13
13
26
7.
(i)
P(AnB)=P(A) + P(8)-P(^uB)=A + A_L=A
(ii)
P{A/B) = P{Ar>B)
4/11
_ 4
5/11
5
P(AnB) _
4/11
_ 2
6/11
" 3
P(B)
P{B/A) =
(iii)
P{A)
We have,
A = {HHH, HTH, THH, TTH}; B = {HHH, HHT}
p{a/b)A
We have,
A = {HHH, HTH, THH, HHT], B = (TTT, TTH, HTT, THT, HHT, THH, HTH]
P{A/B)=y
A = {HHH, HTH, THH, HHT. THT, HTT. TTH}
B = (THH, HTH, HHT, TTH, THT, HTT, TTT}
P{A/B)=^
8.
(i)
(ii)
(iii)
We have, A = {TH, HT}, B = {HT, TH}
P{A/B)=\
We have,
= [HH}. B = (TT}
P{A/B)=0
9.
(i)
(ii)
We have,
^ = {(1,1, 4), (1, 2, 4), (1, 3, 4), (1, 4, 4), (1,5, 4), (1, 6, 4),(2,1, 4), (2, 2, 4), (2, 3, 4),
(2, 4, 4), (2,5, 4), (2, 6, 4),(3,1, 4), (3, 2, 4), (3, 3, 4), (3, 4, 4), (3,5, 4), (3, 6, 4),
(4,1, 4), (4, 2, 4), (4, 3, 4), (4, 4, 4), (4,5, 4), (4, 6, 4), (5,1, 4), (5, 2, 4), (5, 3, 4),
(5, 4, 4), (5,5, 4), (5, 6, 4), (6,1, 4), (6, 2, 4), (6, 3, 4), (6, 4, 4), (6,5, 4), (6,6,4)}
6 = {(6,5,1), (6,5, 2), (6,5, 3), (6,5, 4), (6,5,5), (6,5, 6)}
We observe that n (/4) = 36, n (B) = 6 and
(A n B) = 1
P{AnB) = —,P(A)=—= - and P(B)= —
^
216
216
6
216
= ^,P(B/A) =
6
10.
1
36
P{AnB)
1/216
1
P(A)
“ 1/36 ”6
P{AnB)
1/36
1
Hence, P(A/B) =
P(B)
1/6
12
The sanaple space S is given by S = {MPS, MSP, FSM, FMS, SMF, SFM}
Clearly, A = {MFS, FMS SMF, SFM}, B = {MPS, SPM} and so A n B = {MPS, SPM)
P(AnB) _ 2/6 _1
4/6 ~2
P{AnB)
2/6
= 1 and,
P(B/A) =
P(A/B) =
P(A)
P (B)
2/6
Read

Questions in LaTeX:

\begin{equation}
\begin{aligned}
& \text{30.35} \\
& \text{PROBABILITY} \\
& \text{1} \\
& \text{J_} \\
& \text{6' 36} \\
& \text{(i)} \\
& \text{(ii)} \\
& \text{(iii)} \\
& \text{1} \\
& \text{5} \\
& \text{1} \\
& \text{2} \\
& \text{25. —} \\
& \text{(ii)} \\
& \text{(iii)} \\
& \text{2} \\
& \text{12.} \\
& \text{..} \\
& \text{4} \\
& \text{9.} \\
& \text{(i)} \\
& \text{1} \\
& \text{(ii)} \\
& \text{0} \\
& \text{7.} \\
& \text{1} \\
& \text{11.} \\
& \text{1} \\
& \text{1} \\
& \text{1} \\
& \text{17.} \\
& \text{1} \\
& \text{18.} \\
& \text{5} \\
& \text{25} \\
& \text{2} \\
& \text{1} \\
& \text{4} \\
& \text{5} \\
& \text{2} \\
& \text{22.} \\
& \text{0.12, 0.6} \\
& \text{24.} \\
& \text{0.12, 0.6} \\
& \text{26.} \\
& \text{-} \\
& \text{23.} \\
& \text{10} \\
& \text{7} \\
& \text{8} \\
& \text{6.} \\
& \text{HINTS TO SELECTED PROBLEMS} \\
& \text{24.} \\
& \text{26.} \\
& \text{20.} \\
& \text{1

